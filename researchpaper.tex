\documentclass{article}
\usepackage{listings}
\usepackage{graphicx} 
\usepackage{appendix}
\usepackage{csquotes}
\usepackage{amsmath}
\usepackage[style=apa]{biblatex}
\addbibresource{biblio.bib}
\begin{document}
\thispagestyle{empty}
\begin{center}
\vspace*{4cm}
{\Huge\bfseries Piecewise Functions in Modelling Pricing
Structures\par}
\vspace{1.5cm}
{\large Natalie Hart-Hawco\par}
\vspace{0.5cm}
{\large February 2026\par}
\end{center}


\thispagestyle{empty} 
\newpage
\tableofcontents
\thispagestyle{empty} 
\newpage



\section{Introduction}
In economics, pricing strategies often involve irregular and discontinuous behaviour. As primarily shown by bulk discounts, tax brackets, surge pricing, shipping fees, and certain subscription plans
\textcite{K2017}. Pricing systems cannot always be accurately modelled by continuous functions, as Radic (2024) explains,
\enquote{\textit{Nonlinear pricing is a pricing scheme in which the calculated average price of a unit is a nonlinear function of demand. Typically, it decreases as demand increases.}}
\textcite{radic2024}. Piecewise functions are capable of representing the real-world price changes, as they offer flexibility in their operation.

Piecewise functions appear frequently in first-year calculus courses, where their practicality is often understated. This project aims to bridge that gap by investigating how piecewise functions can be used to model pricing strategies and analyzing the mathematical consequences of said models. Understanding these effects could provide valuable insight into practical applications in economics and business, as real markets depend on threshold-based pricing.

\section{Mathematical Framework}
Piecewise-defined functions are functions whose definitions depend on the value of the input variable \textcite{Stewart2021}. A piecewise function is written as:
\begin{equation*}
f(x) =
\begin{cases}
f_1(x), & x \in D_1, \\
f_2(x), & x \in D_2, \\
\vdots  & \vdots \\
f_n(x), & x \in D_n.
\end{cases}
\end{equation*}

In pricing models, the $x$ will represent the quantity of goods purchased, how many are used, and income. The $f(x)$ will show the total cost paid. 

\subsection{Threshold-Based Pricing}
Looking at continuity at boundary points in mathematical pricing models illustrates a key principle. Defining a price function by:
\begin{equation*}
f(x) =
\begin{cases}
ax, & 0 \le x \le c, \\
bx + d, & x > c,
\end{cases}
\quad a,b,c,d > 0
\end{equation*}

Continuity at $x = c$ requires
\begin{equation*}
\lim_{x \to c^-} f(x) = \lim_{x \to c^+} f(x),
\end{equation*}
which implies
\begin{equation*}
ac = bc + d.
\end{equation*}

Showcases how there is no sudden "bump" in prices when consumption surpasses $c$. Looking from an economic standpoint, discontinuities can put consumers off from increasing demand slightly past a threshold, referred to as threshold avoidance \textcite{Lehner2025}.

\subsection{Differentiability and Marginal Cost}
Continuity will show smooth transitions, while differentiability illustrates marginal price change. If the derivatives on the left and right sides at the boundary point are unequal, the marginal price will abruptly change. 
\[
\lim_{x \to c^-} f'(x) \neq \lim_{x \to c^+} f'(x)
\]

This non-differentiability can be common in real-world pricing and reflects slight changes in pricing rules, seen in bulk discounts and shipping fees. Following \textcite{radic2024}, nonlinear pricing schedules are designed to modify demand elasticity and purchasing incentives.

\subsection{Economic Relation}
Piecewise functions can therefore provide a mathematical framework for analyzing pricing strategies. These functions can show how pricing rules change across intervals and how they affect consumers' purchasing decisions. By examining the continuity and differentiability, we can evaluate whether a pricing system encourages smooth consumption growth or creates artificial barriers at pricing thresholds.

\section{Piecewise Pricing Model}

To reflect a common bulk discount structure, let $x \ge 0$ represent the quantity purchased, and let the total cost function be
$f \colon [0,\infty) \to \mathbb{R}$. represented by
\[
f(x) =
\begin{cases}
10x, & 0 \le x \le 10, \\
8x + 20, & x > 10.
\end{cases}
\]
\subsection{Continuity}
To verify the continuity at $x=10$, we must evaluate the left and right hand side limits. 
\[
\lim_{x \to 10^-} f(x) = 10(10) = 100
\]
and
\[
\lim_{x \to 10^+} f(x) = 8(10) + 20 = 100.
\]

Because both sides are equal, the function is continuous at $x=10$. This continuity ensures that consumers do not experience a sudden price jump when increasing consumption slightly across the threshold. As discussed by \textcite{K2017}, continuity is usually intentionally preserved in nonlinear pricing schemes to avoid discouraging additional purchases.

\subsection{Differentiability and Marginal Pricing}
Although the function is continuous at $x$=10, it is not differentiable as 
\[
f'(x) =
\begin{cases}
10, & 0 < x < 10, \\
8,  & x > 10.
\end{cases}
\]
The left-hand derivative at $x=10$ equals $10$, while the right-hand derivative equals $8$. Since these values are unequal, $f'(10)$ does not exist.

This non-differentiability highlights a kink in the graph of the function. It also represents a discrete reduction in the marginal price once the bulk threshold is exceeded economically. These kinks are a part of nonlinear pricing mechanisms and are frequently used to influence purchasing incentives by rewarding higher consumption levels \textcite{radic2024}.

\subsection{Interpretation}


\section{Graphical Analysis}
To illustrate the behaviour of piecewise pricing, we can showcase the total and marginal costs as quantity functions. Using the same pricing model: 
$f \colon [0,\infty) \to \mathbb{R}$. represented by
\[
f(x) =
\begin{cases}
10x, & 0 \le x \le 10, \\
8x + 20, & x > 10.
\end{cases}
\]
\subsection{Visualization}
\begin{figure}
    \centering
    \includegraphics[width=1.0\linewidth]{Figure_1.png}
    \caption{Piecewise pricing function showing total cost as a function of quantity}
    \label{fig:pricing-graph}
\end{figure}

\subsection{Interpretation of Graph}
In the graph, the left subplot shows $f(x)$ increasing linearly until $x=10$ where a kink appears due to the change in marginal cost, visible in the change in slope from 10 to 8. This reflects the non-differentiability at the threshold, as predicted mathematically. The graphical method clearly demonstrates the impact of piecewise systems on consumer behaviour as it showcases how purchasing slightly more than 10 units is cheaper per unit, providing an incentive for consumers to cross the threshold.

\subsection{Significance}
Graphs help to visualize the mathematical concepts applied in piecewise functions. This method can also be applied to more complex pricing models, including multiple thresholds, nonlinear segments, or discounted health costs.

\section{Multiple Thresholds and Layered Pricing Models}
The pricing model already illustrated focused on a single consumption threshold, which shows local changes in marginal cost and consumer incentives. However, pricing systems can incorporate more than one pricing threshold. For example, mobile data plans, tax brackets, and subscription services, all apply different pricing rules across multiple consumption ranges. As mentioned by \textcite{K2017}, layered pricing structures are a feature of nonlinear pricing strategies, designed to segment consumers based on demand intensity.

\subsection{Multi-Threshold Pricing}
Let $x \ge 0$ be the quantity consumed, and define the total cost function as
\[
f \colon [0,\infty) \to \mathbb{R}
\]
\[
f(x) =
\begin{cases}
a_1 x, & 0 \le x \le c_1, \\
a_2 x + d_1, & c_1 < x \le c_2, \\
a_3 x + d_2, & x > c_2,
\end{cases}
\qquad a_i,\, c_i,\, d_i > 0.
\]
This formulation models a pricing scheme with \textbf{two} consumption thresholds at
\( x = c_1 \) and \( x = c_2 \). These tiered pricing systems are popular in markets where firms attempt to balance revenue extraction with incentives for increased usage \textcite{radic2024}. Continuity of the total cost function requires that

\subsection{Continuity Across Multiple Thresholds}
In layered pricing systems, continuity prevents sudden price jumps at the thresholds. A multi-threshold model requires continuity to hold at both \( x = c_1 \) and \( x = c_2 \) at the same time. Therefore, the conditions
\[
a_1 c_1 = a_2 c_1 + d_1
\quad \text{and} \quad
a_2 c_2 + d_1 = a_3 c_2 + d_2.
\]
must be met to ensure no sudden jumps in prices as consumption crosses each threshold. The intervals define linear segments, resulting in a piecewise-linear function with multiple kinks. As shown in \textcite{K2017}, continuity is often enforced in nonlinear pricing schemes to maintain consumer trust and avoid discouraging marginal increases in demand.




\newpage
\printbibliography
\newpage
\appendix
\section{Python Code for Figure 1}
\label{app:python-code}

\begin{lstlisting}[caption={Python Code for Piecewise Pricing Function}, label={lst:pricing-code}]
import matplotlib.pyplot as plt

# Create x-values
x1 = [x for x in range(0, 11)]
x2 = [x for x in range(10, 21)]

# Define the function 
f1 = [10*x for x in x1]
f2 = [8*x + 20 for x in x2]

# Plot the function
plt.figure()
plt.plot(x1, f1, label=r"$f(x)=10x$")
plt.plot(x2, f2, label=r"$f(x)=8x+20$")

# Mark the breakpoint
plt.scatter([10], [100])
plt.axvline(10, linestyle='--')

plt.xlabel("Quantity")
plt.ylabel("Total cost")
plt.title("Piecewise Pricing Function")
plt.legend()
plt.grid(True)

plt.show()
\end{lstlisting}
\end{document}