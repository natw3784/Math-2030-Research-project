\documentclass[12pt]{article}
\usepackage{csquotes}
\usepackage[style=apa]{biblatex}
\addbibresource{references.bib}

\begin{document}


\section{Introduction}
In economics, pricing strategies often involve discontinuous behaviour. Shown by bulk discounts, tax brackets, surge pricing, shipping fees, and certain subscription plans
\textcite{K2017}. Pricing systems cannot always be accurately measured using continuous functions, as prices can change quickly. Piecewise functions can represent real-world price changes because of the flexibility in their operation. 

Piecewise functions appear in first-year calculus courses, where their practicality isn't emphasized. This project aims to bridge that gap by investigating how piecewise functions can be used to model pricing and analyze mathematical consequences of the models. Understanding these effects could provide insight on the practical applications in economics and business, as real markets depend on jumps in prices.


\section{Research Question}
The primary question this project aims to solve is:

\textbf{How do piecewise-functions influence behaviour in pricing models, and how can mathematical principles be applied?}

Secondary questions include:

\begin{itemize}
    \item{What prices are represented by piecewise functions?}
    \item{How do continuity and differentiability affect piecewise functions?}
    \item{What advantages and limitations are provided by the functions?}
\end{itemize}

\section{Methodology}
    
This project will focus on the continuity and differentiability at boundary points of piecewise functions as well as analytical tools from calculus in order to investigate how changes in pricing rules affect the shape and behaviour of a function.  

\end{document}
