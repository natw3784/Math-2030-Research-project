\documentclass[11pt]{article}
\usepackage{graphicx} 
\usepackage{csquotes}
\usepackage[style=apa]{biblatex}
\addbibresource{references.bib}

\begin{document}
\thispagestyle{empty}
\begin{center}
\vspace*{4cm}
{\Huge\bfseries Piecewise Functions in Modelling Pricing
Structures\par}
\vspace{1.5cm}
{\large Natalie Hart-Hawco\par}
\vspace{0.5cm}
{\large February 2026\par}
\end{center}


\thispagestyle{empty} 
\newpage
\tableofcontents
\thispagestyle{empty} 
\newpage

\section{Introduction}
In economics, pricing strategies often involve irregular and discontinuous behaviour. As primarily shown by bulk discounts, tax brackets, surge pricing, shipping fees, and certain subscription plans
\textcite{K2017}. Pricing systems cannot always be accurately modelled by continuous functions, as Radic (2024) explains,
\enquote{\textit{Nonlinear pricing is a pricing scheme in which the calculated average price of a unit is a nonlinear function of demand. Typically, it decreases as demand increases.}}
\textcite{radic2024}. Piecewise functions are capable of representing the real-world price changes, as they offer flexibility in their operation.

Piecewise functions appear frequently in first-year calculus courses, where their practicality is often understated. This project aims to bridge that gap by investigating how piecewise functions can be used to model pricing strategies and analyzing the mathematical consequences of said models. Understanding these effects could provide valuable insight into practical applications in economics and business, as real markets depend on threshold-based pricing.



\section{Research Question}
The primary question this project aims to solve is:

\textbf{How can piecewise-functions influence behaviour in pricing models, and how can mathematical principles be applied?}

Secondary questions include:

\begin{itemize}
    \item{What pricing systems are represented by piecewise functions?}
    \item{How do continuity and differentiability affect piecewise functions?}
    \item{What advantages and limitations are provided by the functions?}
\end{itemize}

\section{Methodology}
     
This project will examine the continuity and differentiability at boundary
points of piecewise functions, utilizing calculus tools in order to investigate how changes in pricing rules affect the shape and behaviour of a function.

This project will demonstrate pricing structures through the mathematical principles of piecewise functions, providing a real-world application to economics. In addition, Python will be used to visualize and compare different pricing strategies, helping to show how small changes in input variables can affect costs. 
 

\section{Conclusion}

This project is expected to showcase how piecewise functions can be an effective way to model real-world pricing systems with layered costs. Drawing on mathematical properties and examining real-world functions, the proposed study aims to explore the practical implications of piecewise pricing models.

\newpage
\printbibliography

\end{document}
